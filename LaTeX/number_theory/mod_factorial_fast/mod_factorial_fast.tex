\chapter{Поиск факториала по простому модулю за $\O(\sqrt{\min (p, n)} \log n)$}

В прошлом разделе мы научились искать $n! \mod p$ без вхождений $p$ в факториал за $\O(p \log_p n)$ или за $\O(p + \log_p n)$, если заранее предпосчитать все факториалы. Однако чаще всего модуль~--- это число порядка $10^9$, поэтому эти алгоритмы нам не подходят. Давайте научимся искать факториал быстрее.

Давайте сначала научимся искать $n! \mod p$ за $\O(\sqrt{n} \log n)$, если $n < p$.

\begin{observation}
    В зависимости от аккуратности реализации $\log n$ может быть в какой-то степени.
\end{observation}

Обозначим $k = \left\lfloor \sqrt{n} \right\rfloor$. Тогда мы хотим построить алгоритм за $\O(k \log k)$. Пусть $n = k^2 + b$, где $b \le 2 \cdot k$. В таком случае нам достаточно посчитать $\left(k^2\right)!$, а затем за $\O(k)$ домножить его на $b$ недостающих чисел.

Построим многочлен $P(x) := (kx) \cdot (kx - 1) \cdot \ldots \cdot (kx - k + 1)$. Тогда $\left(k^2\right)! = P(1) \cdot P(2) \cdot \ldots \cdot P(k)$. Введем еще один многочлен $Q(x) := P(2x) \cdot P(2x - 1)$. Получается, что $\left(k^2\right)! = Q(1) \cdot Q(2) \cdot \ldots \cdot Q\left(\frac{k}{2}\right)$, если $k$ четно, а если $k$ нечетно, то еще нужно домножить на $P(k)$, но его мы легко можем сделать за $\O(k)$.

Теперь давайте рекурсивно запустимся от многочлена $Q$. Глубина рекурсии будет $\log k$. Однако проблема в том, что изначально многочлен $P$ имел степень $k$, однако многочлен $Q$ имеет степень уже $2k$, и степень будет возрастать. Но если учесть, что многочлен $Q$ нам нужно посчитать только в точках $1, 2, \ldots, \left\lfloor \frac{k}{2} \right\rfloor$, то можно взять вместо $Q$ многочлен $Q \bmod (x - 1) \cdot (x - 2) \cdot \ldots \cdot (x - \left\lfloor \frac{k}{2} \right\rfloor)$. И у такого многочлена уже степень будет $< \left\lfloor \frac{k}{2} \right\rfloor$.

Взятие двух многочленов по модулю можно реализовать за $\O(k \log k)$. Тогда асимптотика алгоритма получается $\O\left(k \log k + \frac{k}{2} \log \left( \frac{k}{2} \right) + \frac{k}{4} \log \left(\frac{k}{4} \right) + \ldots\right) = \O\left(\left(k + \frac{k}{2} + \frac{k}{4} + \ldots\right) \log k\right) = \O(k \log k)$. Что и требовалось.

Чтобы получить алгоритм, который работает при $n \ge p$, нужно просто применить алгоритм из предыдущего раздела, однако вычислять $b!$ с помощью нового метода. $b < p$, поэтому это вычисление будет работать не дольше $\sqrt{p} \log p$. Всего итераций будет $\log_p n$, так что асимптотика получается $\O(\sqrt{p} \log p \cdot \log_p n) = \O\left(\sqrt{p} \log p \frac{\log n}{\log p}\right) = \O\left(\sqrt{p} \log n\right)$. Если объединить случаи $p < n$ и $p \ge n$, получается время работы $\O\left(\sqrt{\min (p, n)} \log n\right)$.

\section{Задачи для практики}

\begin{itemize}
    
    \item \href{https://www.spoj.com/problems/FACTMODP/}{https://www.spoj.com/problems/FACTMODP/}
\end{itemize}

