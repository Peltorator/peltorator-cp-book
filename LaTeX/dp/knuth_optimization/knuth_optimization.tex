\chapter{Оптимизация Кнута~--- Яо} \label{knuth-yao-optimization}

Оптимизация Кнута~--- Яо~--- стандартная оптимизация динамики, позволяющая решать два весьма разных типа динамики (если динамика удовлетворяет некоторому условию, о котором мы поговорим позже):

\begin{itemize}
    \item Задача об оптимальном разбиении массива из $n$ элементов на $k$ подотрезков ($dp[ki][ni] = \min_{i < ni} dp[ki - 1][i] + w(i, ni)$)
    \item Динамика по подотрезкам ($dp[l][r] = w(l, r) + \min_{l < i < r} dp[l][i] + dp[i][r]$)
\end{itemize}

Оба варианта решаются при помощи этой оптимизации за время $O(n^2)$, при том что без оптимизации они решаются за $O(n^2 k)$ и $O(n^3)$ соответственно.

\begin{observation}
    Чаще эта оптимизация называется просто оптимизацией Кнута, однако Кнут придумал, как использовать эту оптимизацию в одном конкретном случае (для построения оптимального двоичного дерева поиска), и его доказательства были слишком сложными. Фрэнсис Яо значительно упростила доказательства Кнута и придумала очень простые условия, гарантирующие корректность этой оптимизации для большого класса других задач.
\end{observation}


Воспринимать материал на какой-то абстрактной задаче может быть тяжело, поэтому давайте сразу сформулируем конкретные примеры задач.

\section{Алгоритм}

\subsection{Задача об оптимальном разбиении массива на $k$ подотрезков}

\begin{problem}
    Дан массив неотрицательных чисел $a$ длины $n$. Назовем мощностью подотрезка квадрат суммы его элементов. Необходимо разбить массив $a$ на $k$ подотрезков таким образом, чтобы минимизировать сумму мощностей подотрезков разбиения. Иными словами, выделить такие $k - 1$ чисел (границы подотрезков разбиения) $i_1$, $i_2$, $\ldots$, $i_{k - 1}$, чтобы

    $$(a_1 + a_2 + \ldots + a_{i_1 - 1})^2 + (a_{i_1} + a_{i_1 + 1} + \ldots + a_{i_2 - 1})^2 + \ldots + (a_{i_{k - 1}} + \ldots + a_{n})^2$$

было минимально.
\end{problem}

В нашем случае $dp[ki][ni]$~--- минимальная стоимость разбиения первых $n$ элементов на $ki$ подотрезков, а $w(i, ni)$~--- мощность подотрезка $(i, ni]$, то есть $(a_{i + 1} + a_{i + 2} + \ldots + a_{ni})^2$. Далее везде мы будем подразумевать, что $w(i, ni)$ можно считать за $O(1)$. Для этого необходимо заранее посчитать префиксные суммы массива $a$ и выразить мощность подотрезка через квадрат разности префиксных сумм его концов.

Базовый алгоритм решает эту задачу за $O(n^2 k)$: мы перебираем количество отрезков по возрастанию, перебираем индекс $ni$, после чего перебираем все возможные индексы $i$ и выбираем среди них оптимальный ответ. Сходу совершенно не ясно, где здесь есть место для улучшения.

Нашу динамику можно представить как таблицу $k \times n$, и $ki$-й ее слой пересчитывается через $ki-1$-й. Тогда давайте заметим, что не обязательно перебирать индексы $ni$ по возрастанию, мы можем перебирать их в любом порядке, в котором нам захочется.

Давайте заметим, что если бы мы для каждой точки $ni$ откуда-то узнали оптимальные точки $i$, через которые нужно пересчитываться, то мы бы могли просто для каждого $ni$ обновить ответ $dp[ki][ni] := dp[ki - 1][i] + w(i, ni)$. Давайте обозначим такое оптимальное $i$ как $opt[ki][ni]$ (если есть несколько индексов $i$, дающих оптимальный ответ, выберем среди них наименьший). Тогда зная $opt[ki][ni]$, мы можем за $O(1)$ найти $dp[ki][ni]$.

Но как же нам найти $opt[ki][ni]$? В этом нам поможет то самое условие, о котором говорилось в начале статьи, которое мы должны наложить на динамику, чтобы оптимизация Кнута~--- Яо работала.

\textbf{Условие монотонности точки разреза по обеим координатам:} Оптимизация Кнута~--- Яо работает тогда и только тогда, когда $opt[ki - 1][ni] \le opt[ki][ni] \le opt[ki][ni + 1]$ для любых $ki$ и $ni$.

Иными словами, при фиксированном $ki$ точка пересчета $opt[ki][ni]$ должна не убывать по $ni$, то есть чем правее $ni$, тем правее левый конец последнего отрезка. А также при фиксированном $ni$ точка пересчета $opt[ki][ni]$ должна не убывать по $ki$, то есть чем больше отрезков мы можем использовать, тем короче должен быть последний отрезок. Иными словами, в матрице $opt$ возрастают и строки, и столбцы. Если подумать, то можно понять, что эти два условия весьма логичны в общем случае, так что часто они действительно выполняются.

О том, почему эти условия выполняются в нашей задаче, мы поговорим позже. Пока что мы просто примем это на веру. На самом деле чаще всего в задачах на оптимизации динамики вы не доказываете, что необходимые условия выполняются, а лишь интуитивно это понимаете или просто верите в это.

Как же эти условия помогут нам решить задачу? Да очень просто! Давайте перебирать количество подотрезков разбиения $ki$ по возрастанию, а индекс $ni$ по убыванию. Тогда в момент, когда мы хотим посчитать $opt[ki][ni]$, уже посчитаны $opt[ki - 1][ni]$ и $opt[ki][ni + 1]$, поэтому ответ нам нужно искать между ними.

Для большего понимания приведем код алгоритма:

\input{dp/knuth_optimization/codes/knuth_segments.cpp}

\begin{observation}
    Заметьте, что так как мы высчитываем все оптимальные точки пересчета, то мы можем при необходимости использовать их для восстановления оптимального разбиения без каких-либо проблем.
\end{observation}

Почему же этот код будет работать за $O(n^2)$? Давайте проанализируем количество итераций, которые мы делаем. Для каждой конкретной пары $ki$, $ni$ мы делаем $opt[ki][ni + 1] - opt[ki - 1][ni] + 1$ операций (мы считаем, что $opt[0][ni] = 0$ и $opt[ki][n] = n - 1$). Интуитивно почти каждое слагаемое $opt[ki][ni]$ побывает в итоговой сумме один раз с плюсом и один раз с минусом и сократится. А если формально, то итоговое время работы~--- это сумма времен работы по всем парам $ki$, $ni$:

$$\sum_{ki = 1}^k \sum_{ni = 0}^{n - 1} (opt[ki][ni + 1] - opt[ki - 1][ni] + 1) =  \sum_{ki = 1}^k \sum_{ni = 0}^{n - 1} opt[ki][ni + 1] - \sum_{ki = 1}^k \sum_{ni = 0}^{n - 1} opt[ki - 1][ni] + \sum_{ki = 1}^k \sum_{ni = 0}^{n - 1} 1 = $$

$$ = \sum_{ki = 1}^k \sum_{ni = 0}^{n - 1} opt[ki][ni + 1] - \sum_{ki = 1}^k \sum_{ni = 0}^{n - 1} opt[ki - 1][ni] + kn$$

Заметим, что почти все $opt[i][j]$ побывают в этой сумме один раз с плюсом и один раз с минусом. Только с плюсом побывают те элементы, для которых $ki - 1 \neq i$ ни для одного $ki$. Это значит, что $i = k$, то есть это последняя строчка матрицы. Кроме того только с плюсом войдут элементы $opt[i][n] = n - 1$, которые формально не принадлежат нашей матрице. Только с минусом аналогично войдут элементы, для которых $ni + 1 \neq j$ ни для какого $ni$. Это значит, что $j = 0$, то есть это первая строка матрицы. Кроме того только с минусом войдут элементы $opt[0][j] = 0$, которые опять же формально не принадлежат нашей матрице.

Если сократить все одинаковые слагаемые, наша асимптотика будет иметь такой вид:

$$kn + \sum_{ni = 0}^{n - 1} opt[k][ni] + \sum_{ki = 1}^k opt[i][n] - \sum_{ki = 1}^{k} opt[ki][0] - \sum_{ni = 0}^{n - 1} opt[0][ni] \le $$

$$ \le kn + n \cdot n + k \cdot n - k \cdot 0 - n \cdot 0 = n \cdot (k + n + k) = O(n^2)$$

Последнее равенство верно, потому что мы разбиваем на $k$ подотрезков, а значит, $k \le n$.

Эту асимптотику можно было понять немного иначе: посмотрим отдельно на каждую диагональ (то есть пары $ki$, $ni$ с фиксированной разностью). Можно легко заметить, что отрезки, по которым пробегается цикл для элементов этой диагонали, пересекаются только по концам, так что для каждой диагонали мы работем за $O(n)$. Так как диагоналей всего $n + k - 1$, то итоговая асимптотика~--- $O(n \cdot (n + k)) = O(n^2)$.

\begin{observation}
    Если вы знакомы с оптимизацией динамики разделяй-и-властвуй (\ref{divide-and-conquer-optimization}), то вы могли заметить, что она решает точно такую же задачу о разбиении на $k$ подотрезков за время $O(n k \log n)$. При этом для разделяй-и-властвуй необходимо только условие монотонности точки разреза по $ni$ (только условие $opt[ki][ni] \le opt[ki][ni + 1]$).

    Поэтому если в задаче применима оптимизация Кнута~--- Яо, то в ней применима и оптимизация разделяй-и-властвуй. Они имеют несравнимые асимптотики, и в зависимости от $k$ одна оптимизация может быть лучше другой. Если $k < \frac{n}{\log n}$, то оптимизация разделяй-и-властвуй будет быстрее, а при $k > \frac{n}{\log n}$ быстрее будет оптимизация Кнута~--- Яо. Однако стоит заметить, что оптимизация Кнута~--- Яо сильно проще в написании (просто три вложенных цикла), а также имеет меньшую константу в асимптотике в силу своей простоты (никакой рекурсии как в разделяй-и-властвуй), так что при возможности стоит использовать именно ее, пока $k$ не значительно меньше $n$.
\end{observation}

\subsection{Динамика по подотрезкам}

Абсолютно аналогично эту же оптимизацию можно применить для задачи динамики по подотрезкам вида

$$dp[l][r] = w(l, r) + \min_{l < i < r} dp[l][i] + dp[i][r]$$

Сразу обратимся к примеру задачи.

\begin{problem}
Есть последовательность из $n$ кучек камней. В $i$-й из них $a_i$ камней. За одну операцию можно объединить две соседние кучки камней в одну, и если в них суммарно было $x$ камней, то эта операция стоит $x$ монет. Необходимо объединить все кучки в одну, заплатив минимальное количество монет.
\end{problem}

Если мы обозначим стоимость объединения всех кучек с $l$-й по $r$-ю (левая граница включительно, а правая не включительно) за $dp[l][r]$, то формула пересчета динамики будет именно такой, какой нужно:

$$dp[l][r] = w(l, r) + \min_{l < i < r} dp[l][i] + dp[i][r]$$

где $w(l, r)$~--- это сумма размеров кучек на полуинтервале от $l$ до $r$. Ответ, соответственно, будет храниться в $dp[0][n]$.

\begin{exercise}
    Опытному читателю, который уже знаком с другими стандартными задачами на оптимизацию Кнута~--- Яо, остается в качестве упражнения понять, почему эта задача, а также задачи построения оптимального двоичного дерева поиска и построения оптимального префиксного кода с сохранением порядка,~--- это на самом деле одни и те же задачи.
\end{exercise}

Обозначим оптимальное $i$ в пересчете динамики за $opt[l][r]$. Тогда утверждается, что опять же выполняется условие монотонности точки разреза по обеим координатам, как и в предыдущей задаче. Значит, мы можем применить оптимизацию Кнута~--- Яо, перебирая левую границу отрезка по возрастанию, а правую~--- по убыванию.

\input{dp/knuth_optimization/codes/knuth_subsegments.cpp}

Так как этот код абсолютно аналогичен коду из предыдущей задачи, то по тем же причинам он работает за $O(n^2)$.

\section{Достаточное условие для монотонности точки разреза}

Как мы уже поняли, необходимыми и достаточными условиями для работы оптимизации Кнута~--- Яо являются условия монотонности точки разреза по обеим координатам, то есть $opt[ki - 1][ni] \le opt[ki][ni] \le opt[ki][ni + 1]$ для любых $ki$ и $ni$. Однако практика показывает, что доказать это утверждение обычно сложно, если не знать, как к нему подступиться. В этом разделе мы познакомимся с достаточным условием на то, чтобы условие монотонности точки разреза выполнялось, которое чаще всего несложно проверить.

Достаточные условия будут немного отличаться для двух разных видов задач, которые мы рассмотрели ранее, поэтому мы посмотрим на них отдельно.

\subsection{Достаточное условие для задачи оптимального разбиения на $k$ подотрезков}

Если наша задача имеет первый вид, то есть

$$dp[ki][ni] = \min_{i < ni} dp[ki - 1][i] + w(i, ni)$$

То необходимым условием для выполнения условия монотонности точки разреза по обеим координатам является неравенство четырехугольника для функции $w$.

\textbf{Неравенство четырехугольника:} Говорят, что функция $w$ удовлетворяет неравенству четырехугольника, если для любых индексов $a \le b \le c \le d$ выполнено неравенство:

$$w(a, c) + w(b, d) \le w(a, d) + w(b, c)$$

Доказать, что утверждение достаточно, не так просто. Прочитать доказательство можно \href{https://cse.hkust.edu.hk/mjg_lib/bibs/DPSu/DPSu.Files/sdarticle_204.pdf}{здесь}, после чего нужно применить лемму 2.2 из \href{https://cse.hkust.edu.hk/mjg_lib/bibs/DPSu/DPSu.Files/p429-yao.pdf}{этой статьи}.
    
\begin{observation}
Обратите внимание, что это условие является достаточным, но не необходимым. То есть если оно выполняется, то оптимизацию Кнута~--- Яо можно применять, однако из того, что оно не выполняется, не следует, что эту оптимизацию применять нельзя.
\end{observation}

\begin{observation}
    Это условие иногда еще называется \textbf{вогнутым} неравенством четырехугольника. Оно достаточно для задачи минимизации. Если же знак в неравенстве стоит в другую сторону, то это \textbf{выпуклое} неравенство четырехугольника, и оно, соответственно, является достаточным для задачи максимизации (представьте, что мы просто заменили функцию $w$ на $-w$, тогда мы перейдем к задаче минимизации с вогнутым неравенством четырехугольника).
\end{observation}

Проверять условие на $w$ намного проще, чем условия на $opt[ni][ki]$, потому что функция $w$ нам уже известна, а значения $opt[ni][ki]$ как-то сложно зависят от динамики.

Безусловно, не во всех задачах неравенство четырехугольника верно, но чаще всего оно все таки выполняется. И если оно верно, то обычно его очень легко проверить. Давайте сделаем это для нашей задачи. В нашем случае $w$~--- это сумма квадратов элементов на отрезке. Давайте обозначим за $x$ сумму элементов на $[a, b)$, за $y$ сумму элементов на $[b, c)$ и за $z$ сумму элементов на $[c, d)$. Тогда неравенство четырехугольника в нашем случае можно переписать как

$$(x + y)^2 + (y + z)^2 \le (x + y + z)^2 + y^2$$

Раскрыв скобки, мы получим

$$x^2 + 2xy + y^2 + y^2 + 2yz + z^2 \le x^2 + 2xy + y^2 + 2yz + z^2 + 2xz + y^2$$

Сокращая одинаковые слагаемые с обеих сторон, мы получим

$$0 \le 2xz$$

Что верно, потому что оба числа $x$ и $z$ неотрицательны (здесь мы как раз пользуемся тем, что элементы изначального массива были неотрицательны).

Тем самым мы доказали, что в нашей задаче выполняется монотонность точки разреза, а значит, наше решение методом Кнута~--- Яо корректно.

Рекомендую в качестве упражнения попытаться воспользоваться этим методом и доказать корректность решения других задач методом Кнута~--- Яо. В большинстве случаев это возможно сделать.

\subsubsection{Ослабленное неравенство четырехугольника}

Кроме того есть \textbf{ослабленное неравенство четырехугольника}, которое говорит, что при $a < c$ верно

$$w(a, c) + w(a + 1, c + 1) \le w(a, c + 1) + w(a + 1, c)$$

Это то же самое неравенство четырехугольника, но $b = a + 1$ и $d = c + 1$. Однако по индукции можно доказать, что из ослабленного неравенства четырехугольника следует обычное неравенство четырехугольника, поэтому на самом деле необходимо проверять только ослабленное. В некоторых задачах это может упростить доказательство. К примеру, в этой:

\begin{problem}
    На дороге расставлены $n$ домов. Необходимо построить $k$ магазинов так, чтобы суммарное расстояние от каждого дома до ближайшего магазина было минимально.
\end{problem}

\subsection{Достаточное условие для задачи динамики по подотрезкам}

Если наша задача имеет второй вид, то есть

$$dp[l][r] = w(l, r) + \min_{l < i < r} dp[l][i] + dp[i][r]$$

То для монотонности точки разреза по обеим координатам необходимы уже два условия: неравенство четырехугольника для $w$, а также \textbf{монотонность} функции $w$.

\textbf{Условие монотонности:} Говорят, что функция $w$ монотонна (по включению), если для любых индексов $a \le b \le c \le d$ выполнено неравенство:

$$w(b, c) \le w(a, d)$$

То есть если один отрезок содержит другой, то у большего отрезка стоимость должна быть не меньше, чем у меньшего.

Это условие интуитивно еще более  логично, чем неравенство четырехугольника, так что почти всегда тоже выполняется.

Доказательство того, что эти два условия достаточны, требует разбора большого количества случаев. Его можно прочитать в \href{https://cse.hkust.edu.hk/mjg_lib/bibs/DPSu/DPSu.Files/p429-yao.pdf}{статье Яо}.

Давайте докажем, что в случае нашей задачи оптимального объединения кучек камней выполнены оба условия.

В нашем случае $w(l, r) = a_l + a_{l + 1} + \ldots + a_{r - 1}$, поэтому на самом деле в неравенстве четырехугольнка достигается равенство, что несложно проверить. Условие монотонности, очевидно, тоже верно, потому что все $a_i$ неотрицательны (ведь это количества камней в кучках). Обратите внимание, что если бы $a_i$ могли быть отрицательны, то условие монотонности бы не выполнялось.

Раз оба условия на $w$ выполнены, то мы можем решать эту задачу при помощи оптимизации Кнута~--- Яо. В качестве упражнения можно потренироваться доказывать эти условия для других задач на динамику по подотрезкам.

\section{Оптимизация Кнута~--- Яо для трудновычислимых стоимостей подотрезков разбиения}

Везде ранее мы подразумевали, что $w(l, r)$ можно вычислить за $O(1)$ для любых индексов $l$ и $r$. Однако это не всегда верно. Бывают ситуации, в которых функция $w$ устроена каким-то сложным образом. В этом случае на помощь может прийти один простой трюк. Даже если функцию $w$ вычислить сложно, бывает нетрудно пересчитать функцию $w$, если подвинуть одну из границ на единицу влево или вправо. Если это возможно, то вычисление каждого нового значения $w$ можно выразить через последовательное движение границ предыдущего посчитанного отрезка к новому. Несложно доказать, что таких движений произойдет суммарно $O(n^2)$, поэтому если передвижение любой границы на единицу в любую сторону можно выполнять за $O(1)$, то асимптотика решения не поменяется.

Подробнее про эту технику можно прочитать в аналогичном разделе в статье про оптимизацию разделяй-и-властвуй (\ref{divide-and-conquer-optimization}).

\section{Источники}

\begin{itemize}
    \item \href{https://cse.hkust.edu.hk/mjg_lib/bibs/DPSu/DPSu.Files/p429-yao.pdf}{Статья Фрэнсис Яо, являющаяся основным источником для этой статьи}

    \item \href{https://cse.hkust.edu.hk/mjg_lib/bibs/DPSu/DPSu.Files/sdarticle_204.pdf}{Статья, в которой представлено доказательство достаточности неравенства четырехугольника для случая оптимального разбиения на $k$ подотрезков}

    \item \href{https://codeforces.com/blog/entry/8219}{Источник информации по разным оптимизациям динамики}
\end{itemize}

\section{Задачи}

\begin{itemize}
    \item \href{https://codeforces.com/group/j9VlExDLmA/contest/100212}{Задача C из этого контеста~--- стандартная задача на построение оптимального префиксного кода с сохранением порядка}
    \item \href{https://codeforces.com/contest/321/problem/E}{Стандартная задача (не забудьте про быстрый ввод)}
    \item \href{https://www.hackerrank.com/contests/ioi-2014-practice-contest-2/challenges/guardians-lunatics-ioi14}{Еще одна задача}
\end{itemize}
